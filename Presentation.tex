\documentclass{beamer}
\usetheme{Montpellier}
\usecolortheme{dolphin}
%% Checking if saving file is working more
%\usepackage{graphicx} %For jpg figure inclusion
%\usepackage{times} %For typeface
%\usepackage{epsfig}
\usepackage{color} %For Comments
\usepackage{beamerthemeshadow} %Paul and Lemmon put this in, take out if you want
%\usepackage[all]{xy}
%\usepackage{float}
%\usepackage{subfigure} 
%\usepackage{hyperref}
%\usepackage{url}
%\usepackage{parskip}
%\usepackage{multirow}

\definecolor{ForestGreen}{RGB}{34,139,34}
\definecolor{BlueViolet}{RGB}{138,43,226}
\definecolor{Coquelicot}{RGB}{255, 56, 0}
\definecolor{Teal}{RGB}{2,132,130}
\definecolor{PrettyBlue}{RGB}{0,0,255}
% Uncomment this if you want to show work-in-progress comments
\newcommand{\comment}[1]{{\bf \tt  {#1}}}
% Uncomment this if you don't want to show comments
%\newcommand{\comment}[1]{}
\newcommand{\emcomment}[1]{\textcolor{ForestGreen}{\comment{Elena: {#1}}}}
\newcommand{\todo}[1]{\textcolor{blue}{\comment{To Do: {#1}}}}
\newcommand{\pscomment}[1]{\textcolor{Coquelicot}{\comment{Paul: {#1}}}}
\newcommand{\mmcomment}[1]{\textcolor{magenta}{\comment{Max: {#1}}}}
\newcommand{\escomment}[1]{\textcolor{BlueViolet}{\comment{Emma: {#1}}}}
\newcommand{\alcomment}[1]{\textcolor{red}{\comment{Lemmon: {#1}}}}
\newcommand{\hfcomment}[1]{\textcolor{Teal}{\comment{Henry: {#1}}}}
%%%%%%%%%%%%%%%%%%%%%%%%%%%%%%%%%%%%%%%%%%

\begin{document}
\title{Restructuring Clojure for a Racket-style Environment}
\date{July 1, 2015}

\begin{frame}
\frametitle {Restructuring Clojure for a Racket-style Environment}
{\centering
\noindent
Henry Fellows, Thomas Hagen, and Elena Machkasova \par

{\it 
HHMI Lunch Presentation\par
July 1, 2015\par}
}
\end{frame}
%frame

\begin{frame}
\frametitle{Table of contents}
\tableofcontents  
\end{frame}

\section{Introduction to the Project}

\begin{frame}
	\frametitle{Racket}
	\begin{itemize}
		\item Currently use Racket
		\item 'toy' language
		\item difficult to make complex projects
		\item Students hitting performance issues
	\end{itemize}
\end{frame}

\begin{frame}
	\frametitle{Why Clojure}
	\begin{itemize}
		\item Used in industry (real life)
		\item Better on Resume
		\item Large community and excellent resources
		\item plethora of libraries (music, graphical)
	\end{itemize}
\end{frame}

\begin{frame}
	\frametitle{What Clojure}
	\begin{itemize}
		\item Clojure is a LISP
		\item Designed by Rich Hickey in 2007
		\item Functional (composition of functions)
		\item Built for Concurrency (simultaneous computation)
		
	\end{itemize}
\end{frame}

\begin{frame}
	\frametitle{Clojure Problems}
	\begin{itemize}
		\item Terrible error messages
		\item Missing graphics libraries for students
		\item Language contains misleading features
		\item Installing the environment is hard.
	\end{itemize}
\end{frame}



\section{Errors}

\begin{frame}
	\frametitle{Error Messages}
	\begin{itemize}
		\item Computers are dumb
		\item Only handle a narrow range of inputs
		\item Primary means of communication
		\item Inherently difficult to create
	\end{itemize}
	\hfcomment{Post office}
\end{frame}

\begin{frame}
	\frametitle{Current Error Messages}
	\begin{itemize}
		\item Incredibly awful
		\item Use strange terminology
		\item Meaningless to most people
		\item Extremely bulky
	\end{itemize}
\end{frame}

\begin{frame}
	\frametitle{New Error Messages}
	\begin{itemize}
		\item Interpret old errors
		\item Replace with new message
		\item Terminology that is friendly to novices
		\item Optional hints to help direct students
	\end{itemize}
\end{frame}

\begin{frame}
	\frametitle{Recent Improvements}
	\begin{itemize}
		\item Changes for Clojure 1.7.0-beta3
		\item Revamped hints to make them more extensible
		\item Made errors for infinite sequences useful \hfcomment{Partially print}
		\item Working on fixing line number reporting
		\item Fixed a large number of smaller issues
	\end{itemize}
\end{frame}

\begin{frame}
	\frametitle{Future Work}
	\begin{itemize}
		\item Look into integrating this with and IDE \hfcomment{Explain}
		\item Spin off our utilities into separate libraries
		\item 
	\end{itemize}
\end{frame}




\section{Quil: Clojure's Graphical Library}

\begin{frame}[fragile]
	\frametitle{What is Quil?}

  		\begin{columns}[t]
		\begin{column}{.55\textwidth}
		\begin{itemize}
  		\item Graphical Library for Clojure
  		\item It can:
  		\begin{itemize}
  	 		\item Draw shapes and images
  	 		\item Move objects on the screen
  	 		\item Make games, pictures, ect..
  		\end{itemize}
  		\end{itemize}
  		 Quil sits on top of Clojure
		\end{column}
		\begin{column}{.45\textwidth}
			\begin{verbatim}
			fun-mode
			^
			Quil
			^
			Clojure
			^
			Java
			\end{verbatim}
		\end{column}
		\end{columns}
\end{frame}





\begin{frame}
	\frametitle{How does Quil work?}
	\begin{itemize}
  		\item Quil takes draw commands
  		%This means you tell quil what to draw and thats it
  	 	\item What you type: (q/rect 500 500 200 200)
  	 	\item What Quil sees: Draw a rectangle at (500, 500)\par 
  	 	and make it 200 pixels wide and 200 pixels tall
  	 	%You can draw many shapes and import images
	 \end{itemize}
\end{frame}

\begin{frame}[fragile]
\frametitle{Quil's fun-mode isn't enough}
	\begin{itemize}
		\item Quil ONLY takes draw commands
		%You can't make a circle template and then use it multiple times
		\item Quil doesn't follow MVC
		%Computer science principle of design
		%Split between your Model, what your information is in
		%Your view, or what you see
		%and your Controller, or what changes things behind the scenes
		%Quil integrates their model and their view, so the thing that holds your information also draws it
		\item Quil code can get confusing and long
		
			
		\begin{columns}[t]
		\begin{column}{.55\textwidth}
		\begin{verbatim}
	(q/fill 80 255 80)
	(q/rect 100 100 50 50)
	(q/no-fill)
	(q/no-stroke)
			\end{verbatim}	
		\item versus
		\begin{verbatim}
	(def lime-rect 
	  (create-rect 50 50 :lime))
	(ds lime-rect 100 100)
		\end{verbatim}
		\end{column}
		\begin{column}{.45\textwidth}
			\begin{figure}[h]
			\includegraphics[width=2cm]{PresentationImages/lime-rectangle.png}
			\end{figure}
		\end{column}
		\end{columns}
		
	\end{itemize}
\end{frame}

\begin{frame}
	\frametitle{Six Squares}
	\begin{itemize}
		\item Especially when you draw more things,\par such as complex shapes
	\end{itemize}
		\begin{figure}[h]
			\includegraphics[width=7cm]{PresentationImages/lime-rectangles.png}
			\centering
		\end{figure}
\end{frame}

\begin{frame}[fragile]
\frametitle{Quil Code}
		\begin{verbatim}
  (let [x 100
  		num 6
  		dist (+ 100 (* (\ num 2) 50))]
	(q/fill 80 255 80)
	(q/rect (- dist (* 1 50)) 100 50 50)
	(q/rect (- dist (* 2 50)) 100 50 50)
	(q/rect (- dist (* 3 50)) 100 50 50)
	(q/rect (- dist (* 4 50)) 100 50 50)
	(q/rect (- dist (* 5 50)) 100 50 50)
	(q/rect (- dist (* 6 50)) 100 50 50))
	(q/no-fill)
	(q/no-stroke)
		\end{verbatim}	

\end{frame}
%long, repetitive, LOTS of numbers and math thats really annoying
%Does NOT scale well, very confusing to look at
%Does NOT conceptually tie to shape
%When you look at this ,you think about math, not your boxes

\begin{frame}[fragile]
\frametitle{Our Code}
	\begin{verbatim}
	(def lime-rect 
	  (create-rect 50 50 :lime))
	  
	(def lime-rectangles 
	  (beside 
	    lime-rect lime-rect lime-rect 
	    lime-rect lime-rect lime-rect))
	  						  
	(ds lime-rectangles 100 100)
	\end{verbatim}
\end{frame}
%Two numbers
%You get to say "Lets put these boxes beside each other"
%You get to draw lime-rectangles, not a much of random things

\begin{frame}
	\frametitle{Our Direction}
	\begin{itemize}
		\item Less paintbrush, more collage
		%An artist finds more use in a brush while a beginner
		%isn't skilled enough
		\item Create shapes, not just draw them
		%Teach students to think about shapes as things
		% Drawing a flower, not 9 cut ellipisies and a circle and 
		%a bunch of coordinates
		\item Easier student code
		%Not distracting from learning basic concepts
		%Works with flow of basic concepts
		\item Give students an idea of how good software should be built              
		%by giving a language like that promotes it by design
	\end{itemize}
\end{frame}

\begin{frame}
	\frametitle{Improvements from Using Quil}
	\begin{itemize}
		\item Experience with language used in the industry
		\item More control for future improvement
		\item Bigger student projects
	\end{itemize}
\end{frame}

\begin{frame}
	\frametitle{Designing super-fun-mode}
	\begin{itemize}
		\item Built on top of Quil
		\item Gives students functions, colors, images, ect..
		\item Allows for easy complex shapes
	\end{itemize}
\end{frame}

\begin{frame}[fragile]
	\frametitle{How super-fun-mode works}
	\begin{itemize}
	\item You start by creating a shape
		\begin{verbatim}
		(def red-square 
		  (create-rect 50 50 :red))
		\end{verbatim}
		\item Note that creating a shape does not draw it
	\begin{columns}[t]
		\begin{column}{.55\textwidth}
		\item From there, you can draw the shape
		\begin{verbatim}
		(ds red-square 500 500)
		\end{verbatim}
		\end{column}
		\begin{column}{.15\textwidth}
		\begin{figure}[h]
			\includegraphics[width=1.6cm]{PresentationImages/red-rectangle.png}
			\end{figure}		
		\end{column}
		\end{columns}
		
	\end{itemize}
\end{frame}

\begin{frame}[fragile]
	\frametitle{How super-fun-mode works}
	\begin{itemize}
	\begin{columns}[t]
		\begin{column}{.55\textwidth}
		\item You can put shapes together to make complex shapes
		\begin{verbatim}
(def rainbow 
    (above red-square 
           orange-square 
           yellow-square 
           green-square 
           blue-square 
           violet-square))
		\end{verbatim}
		\end{column}
		\begin{column}{.3\textwidth}
		\begin{figure}[h]
			\includegraphics[width=1cm]{PresentationImages/rainbow.png}
			\end{figure}		
		\end{column}
		\end{columns}
	\end{itemize}
\end{frame}

\begin{frame}[fragile]
	\frametitle{How super-fun-mode works}
	\begin{itemize}
	\item You can modify the size and orientation of the shape
	\begin{columns}[t]
		\begin{column}{.45\textwidth}
		\begin{verbatim}
		(ds (rotate-shape red-square 45) 
		  500 500)
		\end{verbatim}
		\end{column}
		\begin{column}{.3\textwidth}
		\begin{figure}[h]
			\includegraphics[width=0.8cm]{PresentationImages/red-rectangle-rotate.png}
			\end{figure}		
		\end{column}
		\end{columns} 
		\begin{columns}[t]
		\begin{column}{.45\textwidth}
		\begin{verbatim}
		(ds (scale-shape red-square 2 2) 
		  500 500)
		\end{verbatim}
		\end{column}
		\begin{column}{.3\textwidth}
		\begin{figure}[h]
			\includegraphics[width=1.2cm]{PresentationImages/red-rectangle-scale.png}
			\end{figure}		
		\end{column}
		\end{columns} 
		\begin{columns}[t]
		\begin{column}{.45\textwidth}
		\begin{verbatim}
		(ds (rotate-shape 
      (scale-shape red-square 2 2) 
     45)
  500 500)
		\end{verbatim}
		\end{column}
		\begin{column}{.3\textwidth}
		\begin{figure}[h]
			\includegraphics[width=1.7cm]{PresentationImages/red-rectangle-scale-rotate.png}
			\end{figure}		
		\end{column}
		\end{columns}
	\end{itemize}
\end{frame}
\begin{frame}
	\frametitle{A Few Examples}
	 Please Enjoy a Few Live Examples
\end{frame}

\begin{frame}
	\frametitle{Future Work}
	\begin{itemize}
		\item Fill out more functionality
		\begin{itemize}
			\item Rotate more complex shapes
			\item Pixel-detail Overlay and Overlay-Align
			\item More seemless integration with Quil fun-mode
		\end{itemize}
	\end{itemize}
\end{frame}




\begin{frame}
\frametitle{Acknowledgments}
	Our research was sponsored by:
	\begin{itemize}
	\item HHMI
	\item UMN UROP
	\item UMM MAP
	\end{itemize}
	{\centering
	\noindent
	Thank you! \par
	Any questions? \par
	}
\end{frame}
\end{document}
Status API Training Shop Blog About Help
© 2015 GitHub, Inc. Terms Privacy Security Contact
