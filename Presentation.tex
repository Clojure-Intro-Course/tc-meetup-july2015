\documentclass{beamer}
\usetheme{Montpellier}
\usecolortheme{dolphin}
%% Checking if saving file is working more
%\usepackage{graphicx} %For jpg figure inclusion
%\usepackage{times} %For typeface
%\usepackage{epsfig}
\usepackage{color} %For Comments
\usepackage{beamerthemeshadow} %Paul and Lemmon put this in, take out if you want
%\usepackage[all]{xy}
%\usepackage{float}
%\usepackage{subfigure} 
%\usepackage{hyperref}
%\usepackage{url}
%\usepackage{parskip}
%\usepackage{multirow}

\definecolor{ForestGreen}{RGB}{34,139,34}
\definecolor{BlueViolet}{RGB}{138,43,226}
\definecolor{Coquelicot}{RGB}{255, 56, 0}
\definecolor{Teal}{RGB}{2,132,130}
\definecolor{PrettyBlue}{RGB}{0,0,255}
% Uncomment this if you want to show work-in-progress comments
\newcommand{\comment}[1]{{\bf \tt  {#1}}}
% Uncomment this if you don't want to show comments
%\newcommand{\comment}[1]{}
\newcommand{\emcomment}[1]{\textcolor{ForestGreen}{\comment{Elena: {#1}}}}
\newcommand{\todo}[1]{\textcolor{blue}{\comment{To Do: {#1}}}}
\newcommand{\pscomment}[1]{\textcolor{Coquelicot}{\comment{Paul: {#1}}}}
\newcommand{\mmcomment}[1]{\textcolor{magenta}{\comment{Max: {#1}}}}
\newcommand{\escomment}[1]{\textcolor{BlueViolet}{\comment{Emma: {#1}}}}
\newcommand{\alcomment}[1]{\textcolor{red}{\comment{Lemmon: {#1}}}}
\newcommand{\hfcomment}[1]{\textcolor{Teal}{\comment{Henry: {#1}}}}
%%%%%%%%%%%%%%%%%%%%%%%%%%%%%%%%%%%%%%%%%%

\begin{document}
\title{Restructuring Clojure for a Racket-style Environment}
\date{July 1, 2015}

\begin{frame}
\frametitle {Restructuring Clojure for a Racket-style Environment}
{\centering
\noindent
Henry Fellows, Thomas Hagen, and Elena Machkasova \par

{\it 
HHMI Lunch Presentation\par
July 1, 2015\par}
}
\end{frame}
%frame

\begin{frame}
\frametitle{Table of contents}
\tableofcontents  
\end{frame}

\section{Introduction to the Project}

\begin{frame}
	\frametitle{Problems}
\end{frame}

\begin{frame}
	\frametitle{Why Clojure}
\end{frame}

\begin{frame}
	\frametitle{What Clojure}
\end{frame}

\begin{frame}
	\frametitle{Clojure Problems}
\end{frame}



\section{Errors}

\begin{frame}
	\frametitle{Current Error messages 1}
\end{frame}

\begin{frame}
	\frametitle{Error messages Objectives}
\end{frame}

\begin{frame}
	\frametitle{New Error Messages}
\end{frame}

\begin{frame}
	\frametitle{Recent Improvements}
\end{frame}

\begin{frame}
	\frametitle{Recent Improvements 2}
\end{frame}

\begin{frame}
	\frametitle{Future Work}
\end{frame}

\section{Quil}

\begin{frame}
	\frametitle{What Quil}
\end{frame}

\begin{frame}
	\frametitle{How Quil}
\end{frame}

\begin{frame}
	\frametitle{Why Quil is Bad}
\end{frame}

\begin{frame}
	\frametitle{What we Want and Why}
\end{frame}

\begin{frame}
	\frametitle{Why we Want it}
\end{frame}

\begin{frame}
	\frametitle{SUPERFUNMODE}
\end{frame}

\begin{frame}
	\frametitle{SUPERFUNMODETWO}
\end{frame}

\begin{frame}
	\frametitle{Future Work}
\end{frame}




\begin{frame}
\frametitle{Acknowledgments}
	Our research was sponsored by:
	\begin{itemize}
	\item HHMI
	\item UMN UROP
	\item UMM MAP
	\end{itemize}
	{\centering
	\noindent
	Thank you! \par
	Any questions? \par
	}
\end{frame}
\end{document}
Status API Training Shop Blog About Help
© 2015 GitHub, Inc. Terms Privacy Security Contact
