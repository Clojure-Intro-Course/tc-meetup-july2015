\documentclass{beamer}
\usetheme{Montpellier}
\usecolortheme{dolphin}
%% Checking if saving file is working more
%\usepackage{graphicx} %For jpg figure inclusion
%\usepackage{times} %For typeface
%\usepackage{epsfig}
\usepackage{color} %For Comments
\usepackage{beamerthemeshadow} %Paul and Lemmon put this in, take out if you want
%\usepackage[all]{xy}
%\usepackage{float}
%\usepackage{subfigure} 
%\usepackage{hyperref}
%\usepackage{url}
%\usepackage{parskip}
%\usepackage{multirow}

\definecolor{ForestGreen}{RGB}{34,139,34}
\definecolor{BlueViolet}{RGB}{138,43,226}
\definecolor{Coquelicot}{RGB}{255, 56, 0}
\definecolor{Teal}{RGB}{2,132,130}
\definecolor{PrettyBlue}{RGB}{0,0,255}
% Uncomment this if you want to show work-in-progress comments
\newcommand{\comment}[1]{{\bf \tt  {#1}}}
% Uncomment this if you don't want to show comments
%\newcommand{\comment}[1]{}
\newcommand{\emcomment}[1]{\textcolor{ForestGreen}{\comment{Elena: {#1}}}}
\newcommand{\todo}[1]{\textcolor{blue}{\comment{To Do: {#1}}}}
\newcommand{\pscomment}[1]{\textcolor{Coquelicot}{\comment{Paul: {#1}}}}
\newcommand{\mmcomment}[1]{\textcolor{magenta}{\comment{Max: {#1}}}}
\newcommand{\escomment}[1]{\textcolor{BlueViolet}{\comment{Emma: {#1}}}}
\newcommand{\alcomment}[1]{\textcolor{red}{\comment{Lemmon: {#1}}}}
\newcommand{\hfcomment}[1]{\textcolor{Teal}{\comment{Henry: {#1}}}}
%%%%%%%%%%%%%%%%%%%%%%%%%%%%%%%%%%%%%%%%%%

\begin{document}
\title{Improving Clojure Usablilty for Introductory Course}
\date{July 8, 2015}

\begin{frame}
\frametitle {Improving Clojure Usablilty for Introductory Course}
{\centering
\noindent
Henry Fellows, Thomas Hagen, Sean Stockholm, Ryan McArthur, and Elena Machkasova \par

{\it 
Minnesota Clojure Users Group Meeting\par
July 8, 2015\par}
}
\end{frame}
%frame

\begin{frame}
\frametitle{Table of contents}
\tableofcontents  
\end{frame}

\section{Overview of the Project}

\begin{frame}
	\frametitle{Goals}
	\begin{itemize}
		\item Integrate Clojure into an introductory CS class
		\item Currently use Racket
		\begin{itemize}
			\item Limited teaching language
			\item Difficult to make complex projects
			\item Students hitting performance issues (!)
		\end{itemize}
	\end{itemize}
\end{frame}

\begin{frame}
	\frametitle{Why use Clojure?}
	\begin{itemize}
		\item Used in industry
		\item Better on resume
		\item Support for concurrency
		\item Large community and excellent resources
		\item Excellent libraries (data processing, image recognition, graphical, musical)
	\end{itemize}
\end{frame}

\begin{frame}
	\frametitle{Issues with Clojure}
	\begin{itemize}
		\item Confusing error messages
		\item Lack of beginner-friendly graphics libraries in functional style
		\item Some misleading or confusing core functions ({\tt conj, some}, string functions).   
	\end{itemize}
\end{frame}

\section{Error Handling}

\begin{frame}
	\frametitle{Error Messages}
	\begin{itemize}
		\item Computers are literal
		\item Primary means of communication
		\item Inherently difficult to create
	\end{itemize}
\end{frame}

\begin{frame}[fragile]
\frametitle{Example Native Error}
		\begin{verbatim}
		Exception in thread "main" clojure.lang.ArityException:
		Wrong number of args (3) passed to: core/cons, compiling:
		(/tmp/form-init3025539740275626138.clj:1:72)
	at clojure.lang.Compiler.load(Compiler.java:7142)
	at clojure.lang.Compiler.loadFile(Compiler.java:7086)
	at clojure.main$load_script.invoke(main.clj:274)
	at clojure.main$init_opt.invoke(main.clj:279)
	at clojure.main$initialize.invoke(main.clj:307)
	at clojure.main$null_opt.invoke(main.clj:342)
	at clojure.main$main.doInvoke(main.clj:420)
	at clojure.lang.RestFn.invoke(RestFn.java:421)
	at clojure.lang.Var.invoke(Var.java:383)
	at clojure.lang.AFn.applyToHelper(AFn.java:156)
	at clojure.lang.Var.applyTo(Var.java:700)
	at clojure.main.main(main.java:37)
		\end{verbatim}	
\end{frame}

\begin{frame}
	\frametitle{Current Error Messages}
	\begin{itemize}
		\item Incredibly awful
		\item Use strange terminology
		\item Meaningless to most people
		\item Extremely bulky
	\end{itemize}
\end{frame}

\begin{frame}
	\frametitle{New Error Messages}
	\begin{itemize}
		\item Interpret old errors
		\item Replace with new message
		\item Terminology that is friendly to novices
		\item Consistency within error messages
	\end{itemize}
\end{frame}

\begin{frame}[fragile]
\frametitle{First iteration}
		\begin{verbatim}
Error: Wrong number of arguments (3) passed to a function cons.
Found in file core.clj on line 108 in function -main.
	intro.core/-main (core.clj line 108)
	\end{verbatim}	
\end{frame}

\begin{frame}[fragile]
\frametitle{Current message}
		\begin{verbatim}
Error: You cannot pass three arguments to a function cons, need two.
Found in file core.clj on line 108 in function -main.
	intro.core/-main (core.clj line 108)
	\end{verbatim}	
\end{frame}

\begin{frame}[fragile]
\frametitle{Compilation messages}
	We can handle compiler errors
	\begin{itemize}
	\item Compiler errors are often nested
	\item Many compiler errors in Clojure 1.7.0 were runtime in earlier versions
	\end{itemize}
\end{frame}

\begin{frame}[fragile]
\frametitle{Current message}
	\begin{verbatim}
	Exception in thread "main" java.lang.IllegalArgumentException:
	Parameter declaration "5" should be a vector, compiling:
	(core.clj:104:5)
	at clojure.lang.Compiler.macroexpand1(Compiler.java:6644)
	at clojure.lang.Compiler.analyzeSeq(Compiler.java:6719)
	at clojure.lang.Compiler.analyze(Compiler.java:6524)
	at clojure.lang.Compiler.analyze(Compiler.java:6485)
	at clojure.lang.Compiler$BodyExpr$Parser.parse(Compiler.java:5861)
	at clojure.lang.Compiler$TryExpr$Parser.parse(Compiler.java:2261)
	at clojure.lang.Compiler.analyzeSeq(Compiler.java:6733)
	at clojure.lang.Compiler.analyze(Compiler.java:6524)
	at clojure.lang.Compiler.analyze(Compiler.java:6485)
	at clojure.lang.Compiler$BodyExpr$Parser.parse(Compiler.java:5861)
	at clojure.lang.Compiler$FnMethod.parse(Compiler.java:5296)
	at clojure.lang.Compiler$FnExpr.parse(Compiler.java:3925)
	\end{verbatim}	
\end{frame}

\begin{frame}[fragile]
\frametitle{Current message}
		\begin{verbatim}
Error: Parameters for defn must be a vector, but 5 was found instead.
Found in file core.clj on line 107 in function -main.
	clojure.core/assert-valid-fdecl (core.clj line 7180)
	clojure.core/map (core.clj line 2622)
	clojure.core/seq (core.clj line 135)
	clojure.core/filter (core.clj line 2677)
	clojure.core/seq (core.clj line 135)
	clojure.core/assert-valid-fdecl (core.clj line 7184)
	clojure.core/sigs (core.clj line 225)
	clojure.core/defn (core.clj line 303)
	intro.core/-main (core.clj line 107)
		\end{verbatim}	
\end{frame}

\begin{frame}
	\frametitle{Recent Improvements}
	\begin{itemize}
		\item Fixed issues with arity 
		\item Made errors for lazy sequences useful
		\item \hfcomment{Expand?}
		\item Working on fixing line number reporting
		\item Fixed a large number of smaller issues
	\end{itemize}
\end{frame}

\section{Clojure's Graphical Library}

\begin{frame}[fragile]
	\frametitle{What is Quil?}

  		\begin{columns}[t]
		\begin{column}{.55\textwidth}
		\begin{itemize}
  		\item Graphical Library for Clojure
  		\item It can:
  		\begin{itemize}
  	 		\item Draw shapes and images
  	 		\item Move objects on the screen
  	 		\item Make games, pictures, ect..
  		\end{itemize}
  		\end{itemize}
		\end{column}
		\begin{column}{.45\textwidth}
			\begin{verbatim}
			fun-mode
			^
			Quil
			^
			Clojure
			^
			Java
			\end{verbatim}
		\end{column}
		\end{columns}
\end{frame}

\begin{frame}[fragile]
\frametitle{Quil's fun-mode isn't enough}
	\begin{itemize}
		\item Quil ONLY takes draw commands
		%You can't make a circle template and then use it multiple times
		\item Quil doesn't separate the model from the view
		%Computer science principle of design
		%Quil integrates their model and their view, so the thing that holds your information also draws it
		%Visualizing a rectangle in the abstract vs in your head
		\item Quil code can get confusing and long
		
			
		\begin{columns}[t]
		\begin{column}{.55\textwidth}
		\begin{verbatim}
	(q/fill 80 255 80)
	(q/rect 100 100 50 50)
	(q/no-fill)
	(q/no-stroke)
			\end{verbatim}
		\end{column}
		\begin{column}{.45\textwidth}
			\begin{figure}[h]
			\includegraphics[width=2cm]{PresentationImages/lime-rectangle.png}
			\end{figure}
		\end{column}
		\end{columns}
		
	\end{itemize}
\end{frame}

\begin{frame}[fragile]
	\frametitle{Designing super-fun-mode}
	\begin{itemize}
	%Cat picture here
	\begin{columns}[t]
		\begin{column}{.55\textwidth}
		\item Built on top of fun-mode
		\item Gives students functions, colors, images, ect..
		\item Easy to read and change program code
		\item Allows for easy complex shapes
		\end{column}
		\begin{column}{.45\textwidth}
			\begin{verbatim}
			super-fun-mode
			^
			fun-mode
			^
			Quil
			^
			Clojure
			^
			Java
			\end{verbatim}
		\end{column}
		\end{columns}
	\end{itemize}
\end{frame}

\begin{frame}[fragile]
	\frametitle{How super-fun-mode works}
	\begin{itemize}
	\item You start by creating a shape
		\begin{verbatim}
		(def red-square 
		  (create-rect 50 50 :red))
		\end{verbatim}
		\item Note that creating a shape does not draw it
	\begin{columns}[t]
		\begin{column}{.55\textwidth}
		\item From there, you can draw the shape
		\begin{verbatim}
		(draw-shape red-square 500 500)
		\end{verbatim}
		\end{column}
		\begin{column}{.15\textwidth}
		\begin{figure}[h]
			\includegraphics[width=1.6cm]{PresentationImages/red-rectangle.png}
			\end{figure}		
		\end{column}
		\end{columns}
		
	\end{itemize}
\end{frame}

\begin{frame}
	\frametitle{How super-fun-mode works technically}
	\begin{itemize}
	\item Underneath, super-fun-mode builds a hashmap or a vector of hashmaps (in the case of complex shapes) with holds relevant information including:
	\begin{itemize}
		\item The shape's width and height
		\item The complex shape's width and height
		\item The rotation angle of the shape
		\item The function to draw the shape
		\end{itemize}
		
	\end{itemize}
\end{frame}



\begin{frame}[fragile]
	\frametitle{super-fun-mode complex shapes}
	\begin{itemize}
	\begin{columns}[t]
		\begin{column}{.55\textwidth}
		\item You can put shapes together to make complex shapes
		\begin{verbatim}
(def tower 
    (above red-square 
           orange-square 
           yellow-square 
           green-square 
           blue-square 
           violet-square))
		\end{verbatim}
		\end{column}
		\begin{column}{.3\textwidth}
		\begin{figure}[h]
			\includegraphics[width=1cm]{PresentationImages/rainbow.png}
			\end{figure}		
		\end{column}
		\end{columns}
	\end{itemize}
\end{frame}

\begin{frame}
	\frametitle{Six squares}
	\begin{itemize}
		\item The difference becomes quite apparent with complexity 
	\end{itemize}
		\begin{figure}[h]
			\includegraphics[width=7cm]{PresentationImages/lime-rectangles.png}
			\centering
		\end{figure}
\end{frame}

\begin{frame}[fragile]
\frametitle{Quil code}
		\begin{verbatim}
(let [x 100
  		   numb 6
  		   dist (+ 100 (* (\ numb 2) 50))]
	  (q/fill 80 255 80)
	  (q/rect (- dist (* 1 50)) 100 50 50)
	  (q/rect (- dist (* 2 50)) 100 50 50)
	  (q/rect (- dist (* 3 50)) 100 50 50)
	  (q/rect (- dist (* 4 50)) 100 50 50)
	  (q/rect (- dist (* 5 50)) 100 50 50)
	  (q/rect (- dist (* 6 50)) 100 50 50))
	(q/no-fill)
		\end{verbatim}	

\end{frame}
%long, repetitive, LOTS of numbers and math thats really annoying
%Does NOT scale well, very confusing to look at
%Does NOT conceptually tie to shape
%When you look at this ,you think about math, not your boxes

\begin{frame}[fragile]
\frametitle{Our code}
	\begin{verbatim}
	(def lime-rect 
	  (create-rect 50 50 :lime))
	  
	(def lime-rectangles 
	  (beside 
	    lime-rect lime-rect lime-rect 
	    lime-rect lime-rect lime-rect))
	  						  
	\end{verbatim}
\end{frame}
%Two numbers
%You get to say "Lets put these boxes beside each other"
%You get to draw lime-rectangles, not a bunch of random things


\begin{frame}[fragile]
	\frametitle{Rotation and scaling}
	\begin{itemize}
	\item You can modify the size and orientation of the shape
	\begin{columns}[t]
		\begin{column}{.45\textwidth}
		\begin{verbatim}
		(rotate-shape red-square 45)
		\end{verbatim}
		\end{column}
		\begin{column}{.3\textwidth}
		\begin{figure}[h]
			\includegraphics[width=0.8cm]{PresentationImages/red-rectangle-rotate.png}
			\end{figure}		
		\end{column}
		\end{columns} 
		\begin{columns}[t]
		\begin{column}{.45\textwidth}
		\begin{verbatim}
		(scale-shape red-square 2 2)
		\end{verbatim}
		\end{column}
		\begin{column}{.3\textwidth}
		\begin{figure}[h]
			\includegraphics[width=1.2cm]{PresentationImages/red-rectangle-scale.png}
			\end{figure}		
		\end{column}
		\end{columns} 
		\begin{columns}[t]
		\begin{column}{.45\textwidth}
		\begin{verbatim}
		(rotate-shape 
		  (scale-shape red-square 2 2)
		 45)

		\end{verbatim}
		\end{column}
		\begin{column}{.3\textwidth}
		\begin{figure}[h]
			\includegraphics[width=1.7cm]{PresentationImages/red-rectangle-scale-rotate.png}
			\end{figure}		
		\end{column}
		\end{columns}
	\end{itemize}
\end{frame}



\begin{frame}[fragile]
	\frametitle{Overlaying and complex shapes}
	\begin{itemize}
	\item You can put shapes on top of each other 
	\begin{columns}[t]
		\begin{column}{.45\textwidth}
		\begin{verbatim}
		(overlay window roof)
		\end{verbatim}
		\end{column}
		\begin{column}{.3\textwidth}
		\begin{figure}[h]
			\includegraphics[width=1.22cm]{PresentationImages/roof.png}
			\end{figure}		
		\end{column}
		\end{columns} 
		\begin{columns}[t]
		\begin{column}{.45\textwidth}
		\begin{verbatim}
		(overlay-align :bottom :center 
		     door 
		     red-rect)
		\end{verbatim}
		\end{column}
		\begin{column}{.3\textwidth}
		\begin{figure}[h]
			\includegraphics[width=1.2cm]{PresentationImages/body.png}
			\end{figure}		
		\end{column}
		\end{columns} 
		\begin{columns}[t]
		\begin{column}{.45\textwidth}
		\begin{verbatim}
		(scale-shape 
		  (above (overlay top bottom)) 
		1.4 1.4)

		\end{verbatim}
		\end{column}
		\begin{column}{.3\textwidth}
		\begin{figure}[h]
			\includegraphics[width=2cm]{PresentationImages/house.png}
			\end{figure}		
		\end{column}
		\end{columns}
	\end{itemize}
	
\end{frame}

\begin{frame}[fragile]
	\frametitle{Other complex functions}
	\begin{itemize}
	\item You can orient your besides and aboves as well
	\begin{columns}[t]
		\begin{column}{.45\textwidth}
		\begin{verbatim}
(beside-align :top 
              tower 
              tight-rope 
              tower)
		\end{verbatim}
		\end{column}
		\begin{column}{.3\textwidth}
		\begin{figure}[h]
			\includegraphics[width=1.22cm]{PresentationImages/towers.png}
			\end{figure}		
		\end{column}
		\end{columns} 
		\begin{columns}[t]
		\begin{column}{.45\textwidth}
		\begin{verbatim}
(above-align :right 
             block1 
             block1.3
             block1.6)
		\end{verbatim}
		\end{column}
		\begin{column}{.3\textwidth}
		\begin{figure}[h]
			\includegraphics[width=1.1cm]{PresentationImages/left-tower.png}
			\end{figure}		
		\end{column}
		\end{columns} 
		\begin{columns}[t]
		\begin{column}{.45\textwidth}
		\begin{verbatim}
(beside-align :top 
              tower-aligned-R 
              tight-rope 
              tower-aligned-L)
		\end{verbatim}
		\end{column}
		\begin{column}{.3\textwidth}
		\begin{figure}[h]
			\includegraphics[width=1.5cm]{PresentationImages/two-tower.png}
			\end{figure}		
		\end{column}
		\end{columns}
	\end{itemize}
	
\end{frame}



%Build house slides, round window

\begin{frame}
	\frametitle{Our direction}
	\begin{itemize}
		\item Less paintbrush, more collage
		%An artist finds more use in a brush while a beginner
		%isn't skilled enough
		\item Create shapes, not just draw them
		%Teach students to think about shapes without positions
		% Drawing a flower, not 9 cut ellipisies and a circle and 
		%a bunch of coordinates
		\item Easier student code
		%Not distracting from learning basic concepts
		%Works with flow of basic concepts
		\item Give students an idea of how good software should be built              
		%by giving a language like that promotes it by design
	\end{itemize}
\end{frame}

\begin{frame}
	\frametitle{A few examples}
	 Please Enjoy a Few Live Examples
\end{frame}

\begin{frame}
	\frametitle{Future work}
	\begin{itemize}
		\item Fill out more functionality
		\begin{itemize}
			\item Rotate more complex shapes
			\item Pixel-detail Overlay and Overlay-Align
			\item More seemless integration with Quil fun-mode
		\end{itemize}
		\item Open Source the project
		\item Integrate a Clojure sound library
	\end{itemize}
\end{frame}

\begin{frame}
\frametitle{Acknowledgments}
	Our research was sponsored by:
	\begin{itemize}
	\item HHMI
	\item LSAMP
	\end{itemize}
	{\centering
	\noindent
	Thank you! \par
	Any questions? \par
	}
\end{frame}
\end{document}
Status API Training Shop Blog About Help
© 2015 GitHub, Inc. Terms Privacy Security Contact
